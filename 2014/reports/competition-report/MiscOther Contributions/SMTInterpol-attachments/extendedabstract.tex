\documentclass[11pt,a4paper]{easychair}
\usepackage[english]{babel}
\usepackage{xspace}
\hypersetup{colorlinks,linkcolor=blue,filecolor=black,urlcolor=blue,citecolor=black}

\newcommand\si{SMTInterpol\xspace}
\newcommand\T{$\mathit{T}$\xspace}
\newcommand\version{2.0\xspace}

\newtheorem{example}{Example}

\title{\si\\{\large Version \version}}

\author{J\"urgen Christ \qquad Jochen Hoenicke \qquad Alexander Nutz\\
  University of Freiburg\\
  \texttt{\{christj,hoenicke,nutz\}@informatik.uni-freiburg.de}}

\titlerunning{\si}
\authorrunning{Christ, Hoenicke, and Nutz}

\begin{document}
\maketitle
\section*{Introduction}
\si~\cite{smtinterpol} is a proof-producing and interpolating SMT-solver written in Java.  
It is available from 
\url{http://ultimate.informatik.uni-freiburg.de/smtinterpol}
under the GNU Lesser General Public License (LGPL) version~3.0.
The solver reads input in SMTLIB format.  It includes a parser for version
1.2, and a parser for the current version.  All required and some optional
commands of the SMTLIB standard are supported.
\si supports the quantifier-free combination of uninterpreted functions and
linear (real and integer) arithmetic, i.\,e., the SMTLIB logics QF\_UF,
QF\_LIA, QF\_LRA, QF\_UFLIA, and QF\_UFLRA.
For all these logics,
\si supports the computation of inductive sequences of Craig interpolants,
which are used by 
several interpolation-based model
checkers~\cite{HHP09,HHP10,DBLP:conf/vmcai/ErmisHP12}.

All formulas are stored in a central term repository.  The repository
type-checks the formulas.  Asserted formulas are converted to CNF using
Plaisted--Greenbaum encoding~\cite{DBLP:journals/jsc/PlaistedG86}.  The core
of the solver is a CDCL engine that is connected to multiple theories.  The
engine uses these theories during constraint propagation, backtracking, and
consistency checking.

For uninterpreted functions and predicates, we use a theory solver based on
the congruence closure algorithm.  An extension to arrays and quantifiers via
e-matching is under development.  For linear arithmetic, we use a theory
solver based on the Simplex algorithm~\cite{DBLP:conf/cav/DutertreM06}.  It
always computes the strongest bounds that can be derived for a variable and
uses them during satisfiability checks.  If a conflict cannot be explained
using known literals, the solver derives new literals and uses them in
conflict explanation.  Disequalities are used to
strengthen bounds, or are delayed until final checks.  The
solver supports integer arithmetic using a variant of the cuts from proof
technique~\cite{DBLP:conf/cav/DilligDA09} together with a branch-and-bound
engine.

\si uses a variant of model-based theory
combination~\cite{DBLP:journals/entcs/MouraB08}.  The linear arithmetic solver
does not propagate equalities between shared variables but introduces them as
decision points.  The model mutation algorithm resolves disequalities
and tries to create as many distinct equivalence classes as possible.

\section*{Interpolation}

\si produces inductive sequences of interpolants for the SMTLIB logics QF\_UF,
QF\_LRA, QF\_UFLRA, QF\_LIA, and QF\_UFLIA.  Since the integer logics defined
in the SMTLIB standard are not closed under interpolation, \si extends these
logics with the division and modulo operators with constant divisor.

The architecture of the interpolation engine roughly follows the DPLL(\T)
paradigm: A \emph{core interpolator} produces \emph{partial interpolants} for
the resolution steps while theory specific interpolators produce partial
interpolants for \T-lemmas.  In the presence of \emph{mixed literals}, i.e.,
literals that use symbols from more than one block of the interpolation
problem,  an approach loosely based on the method of Yorsh et
al.~\cite{DBLP:conf/cade/YorshM05} is used.  The basic idea of the approach used in \si
is to virtually purify each mixed literal using an auxiliary variable, to
restrict the places where the variable may occur in partial interpolants, and
to use special resolution rules to eliminate the variable when the mixed
literal is used as a pivot.  In essence, for convex theories, this approach
can be seen as a lazy version of the method of Yorsh et al.  The approach also
works for non-convex theories using disjunctions in the interpolants.  

\section*{New Developments}
Compared to the version that participated in the SMT competition in 2011,
several performance improvements have been implemented.  These include clause
minimization techniques and a new pivot strategy in the linear arithmetic
solver.  Additionally, the assertion stack management has been reworked to be
more stable.

%What we can do with proofs
For unsatisfiable formulas, \si supports the optional SMTLIB command
\verb+get-unsat-core+.  This command extracts an unsatisfiable core from a
proof tree.  This technique does not guarantee minimality of the returned
core.  The optional command \verb+get-proof+ was already supported in last
year's version.  Additionally, the non-standard command
\verb+get-interpolants+ can be used to compute an inductive sequence of
Craig interpolants.  The interpolation engine is complete for all logics
supported by \si.

%Models
For satisfiable formulas, \si supports the optional SMTLIB command
\verb+get-value+ and the non-standard command \verb+get-model+.  These
commands can be used to inspect the model produced by \si.  For uninterpreted
sorts, \si generates a finite sort interpretation.  The domain of this
interpretation contains input terms instead of abstract values (see the SMTLIB
standard).

\bibliography{sysdec}
\bibliographystyle{alpha}
\end{document}
